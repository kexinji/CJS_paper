% Document settings
\documentclass[12pt]{article}
\usepackage[margin=1in]{geometry}
\usepackage[pdftex]{graphicx}
\usepackage{multirow}
\usepackage{setspace}
\usepackage{hyperref}

\usepackage{color}

\usepackage{amsmath,amssymb,amstext} 

\pagestyle{plain}



\begin{document}

{\large \bf Univariate simulation}

For fixed regression parameter $\beta_1$, non-parametric function $f$, variance components  
$\boldsymbol  \theta = (\phi, \sigma^2, \theta_2, \theta_3)$, 
we generate the covariate $X$, the random intercept $b_i$, the OU process $U_i$ and the measurement error $\epsilon$ with the assumption that all observation time points are the same for all subjects $t_i = t$. 
Denote $Y_{ij}$ to be the univariate hormone response for subject $i$ at time point $j$,  we simulate the longitudinal response data according to the model 
$$
Y_{ij} = \beta_1 {\rm age}_i + f(t_{i}) + b_i + U_i(t_{ij}) + \epsilon_{ij}. 
$$
where the nonparametric periodic function of time is generated from the sine function 
$
f(t) = 5  \sin(\pi t/15)
$
with period length 30;  the $b_i$ are independent random intercepts following a normal distribution with mean zero and variance $\phi$; the mean-zero Gaussian process $U_i$ modelling the serial correlation are simulated from stationary OU process with covariance function ${\mathrm cov}(U_i (s), U_i(t)) = {\theta_3^2 {\mathrm exp}\{-\theta_2|s - t|\} / 2\theta_2}$, and the independent measurement errors $\epsilon_i$ are from Normal distribution with mean zero and variance $\sigma^2$. The simulated response is of one cycle for now.



The variance components $\bf \theta = (\phi, \sigma^2, \theta_2, \theta_3)$ and the smoothing parameter $\lambda$ will be estimated using Fisher-Scoring algorithm. 




\end{document}



